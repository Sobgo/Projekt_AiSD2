\documentclass{article}

\usepackage[T1]{fontenc}
\usepackage[utf8]{inputenc}
\usepackage[polish]{babel}

\usepackage{csquotes}

\usepackage{hyperref}
\usepackage[capitalize,nameinlink]{cleveref}
\usepackage{url}

\usepackage[dvipsnames]{xcolor}
\hypersetup{
colorlinks=true,
linkcolor=BrickRed,
citecolor=Green,
urlcolor=blue,
frenchlinks=true,
pdftitle={Atak z trzeciego wymiaru},
pdfpagemode=FullScreen,
}

\usepackage[backend=biber, sorting=ynt]{biblatex}
\addbibresource{bibliography.bib}

\title{Atak z trzeciego wymiaru \\\large Sprawozdanie z projektu}
\author{Mikołaj Juda \and Mateusz Sobkowiak \and Paulina Grabowska}
\date{2024}

\begin{document}
\maketitle

\section{Architektura}
Problemy z wymagań projektu podzieliliśmy na podproblemy, w większości
rozwiązywalne istniejącymi algorytmami.
Rozwiązanie każdego podproblemu jest zaimplementowane jako biblioteka,
która jest wykorzystywana przez program zapewniający interfejs użytkownika
i minimalną logikę łączącą podproblemy. Pozwala to na proste testowanie
i umożliwia łatwe wprowadzenie wielu interfejsów, na przykład
tekstowy interfejs do użycia przez zewnętrzne programy i interfejs graficzny dla człowieka.

\section{Przepływ pracy}
Do organizacji pracy wykorzystywaliśmy narzędzia dostępne w serwisie GitHub.

\noindent Przydzielanie i śledzenie zadań realizowaliśmy na tablicy Kanban w GitHub Projects.

\noindent Kod rozwiązujący podproblemy był rozwijany na osobnych gałęziach,
i łączony z główną gałęzią używając pull requestów po recenzji przez wszystkich członków zespołu.
Używaliśmy też GitHub Actions do automatycznego uruchamiania lintera i testów.

\section{Rozwiązania problemów}

\subsection{Budowa płotu}

\subsubsection{Długość płotu}
Do wyznaczania otoczki wypukłej wykorzystaliśmy algorytm Grahama.

\noindent Testy jednostkowe zawierają sprawdzenie poprawności wyniku dla przypadków krańcowych
i dla danych losowych w oparciu o sprawdzenie wypukłości i zawierania wszystkich punktów
używając pomocniczych funkcji.
\subsubsection{Parowanie tragarzy}
Parowanie tragarzy zrealizowaliśmy jako znajdowanie największego skojarzenia w grafie dwudzielnym.
Rozwiązanie zaimplementowaliśmy naiwnie używając algorytmu Edmondsa-Karpa ze względu na prostotę implementacji.
Przy złożoności \(\mathcal{O}(|V||E|)\)\cite{cs6820matchingnotes}
i fakcie, że algorytm uruchamiany jest tylko raz na początku planowania budowy,
uznaliśmy, że nasze rozwiązanie jest wystarczająco wydajne,
mimo istnienia algorytmów o lepszej złożoności obliczeniowej (np. Hopcrofta-Karpa).

\noindent Testy jednostkowe sprawdzają jedynie czy wynik jest maksymalny,
gdyż nie ma prostej metody na zweryfikowanie czy jest największy
bez obliczania rozwiązania problemu.
\subsubsection{Wyznaczanie dróg}
Wyznaczanie najkrótszych dróg z fabryki do miejsc budowy płotu zrealizowaliśmy
jako znajdowanie najkrótszej ścieżki z pojedynczego źródła
i zaimplementowaliśmy używając algorytmu Dijkstry z implementacją kolejki na kopcu.
Jest to klasyczny sposób rozwiązania tego problemu i dla rozważanych rozmiarów danych
jest niewiele mniej wydajny od bardziej skomplikowanych rozwiązań
o lepszej złożoności obliczeniowej.

\noindent Testy jednostkowe sprawdzają poprawność wyniku
na podstawie porównania długości ścieżek z wcześniej wyznaczonymi poprawnymi wartościami
i przejściu wyznaczonymi ścieżkami sprawdzając prawidłowość.

\subsection{Zapisywanie melodii}
\subsubsection{Zamiana fragmentów}
Wyszukiwanie słów do zamiany w tekście zaimplementowaliśmy algorytmem Aho-Corasicka
ze względu na możliwość wyszukiwania wielu wzorców jednocześnie.
Ma także najlepszą złożoność obliczeniową spośród algorytmów do tego zdolnych.

\noindent Testy jednostkowe sprawdzają poprawność wyniku
dla przypadków prostych, granicznych i losowych na podstawie porównania
z wynikiem algorytmu naiwnego.
\subsubsection{Kompresja}
Do kompresji wykorzystaliśmy kodowanie Huffmana ze względu na prostotę implementacji.
Jest to rozwiązanie wystarczająco wydajne
do jednorazowego zapisywania danych w celach archiwalnych.

\noindent Testy jednostkowe sprawdzają poprawność kompresji i dekompresji
przykładowych tekstów, tekstów losowych oraz przypadków granicznych.

\subsection{Patrole strażników}
Znajdowanie optymalnej trasy dla strażników, która minimializuje ilość odsłuchań melodii
zaimplementowaliśmy algorytmem dynamicznym.
\subsubsection{Dowód poprawności}
Funkcja opiera się o stan \(dp[i]\), który reprezentuje minimalną liczbę odsłuchań
melodii potrzebną do dotarcia do punktu \(i\). Początkowo dla \(dp[0]\) jest on ustawiony na 0,
bo zaczynamy w punkcie 0 bez odsłuchań melodii.
Dla każdego punktu \(i\) funkcja sprawdza możliwe punkty \(j\), do których można dotrzeć z punktu \(i\).
Poziom jasności punktu \(j\) powoduje aktualizację \(dp[j]\) zgodnie z warunkami. Za pomocą tablicy \(previous\)
śledzone są poprzednie punkty na optymalnej ścieżce. Każdy krok wstecz prowadzi do punktu, z którego
przybyliśmy, co gwarantuje, że odtworzona ścieżka jest zgodna z minimalnymi odsłuchami melodii
zapisanymi w \(dp\). Zaimplementowana w ten sposób funkcja gwarantuje znalezienie optymalnego rozwiązania
przy niskiej złożoności obliczeniowej \(\mathcal{O}(n \cdot maxBeforeStop)\) i pamięciowej \(\mathcal{O}(n)\),
gdzie \(n\) to ilość punktów do obejścia, a \(maxbeforeStop\) to
maksymalna liczba punktów po których strażnik musi się zatrzymać.

\noindent Testy jednostkowe sprawdzają poprawność działania algorytmu porównując wynik
do wcześniej ustalonego poprawnego wyniku oraz do wyniku algorytmu naiwnego.
\section{Dokumentacja}
Dokumentacja jest generowana przez Doxygen na podstawie komentarzy w kodzie.

\printbibliography[heading=bibnumbered]
\end{document}
