\documentclass{article}

\usepackage[T1]{fontenc}
\usepackage[utf8]{inputenc}
\usepackage[polish]{babel}

\usepackage{csquotes}

\usepackage{hyperref}
\usepackage[capitalize,nameinlink]{cleveref}
\usepackage{url}

\usepackage[dvipsnames]{xcolor}
\hypersetup{
colorlinks=true,
linkcolor=BrickRed,
citecolor=Green,
urlcolor=blue,
frenchlinks=true,
pdftitle={Atak z trzeciego wymiaru},
pdfpagemode=FullScreen,
}

\usepackage[backend=biber, sorting=ynt]{biblatex}
\addbibresource{bibliography.bib}

\title{Atak z trzeciego wymiaru \\\large Sprawozdanie z projektu}
\author{Mikołaj Juda \and Mateusz Sobkowiak \and Paulina Grabowska}
\date{2024}

\begin{document}
\maketitle
\section{Przepływ pracy}
aaaaaaaaaaaaaaaa
\section{Rozwiązania problemów}
\subsection{Długość płotu}
aaaaaaaaaaaaaaaa
\subsection{Budowa płotu}
\subsubsection{Parowanie tragarzy}
Parowanie tragarzy zrealizowaliśmy jako znajdowanie największego skojarzenia w grafie dwudzielnym.
Rozwiązanie zaimplementowaliśmy naiwnie używając algorytmu Edmondsa-Karpa ze względu na prostotę implementacji
(wykorzystaliśmy implementację, którą napisaliśmy wcześniej). Przy złożoności \(\mathcal{O}(|V||E|)\)\cite{cs6820matchingnotes}
i fakcie, że algorytm uruchamiany jest tylko raz na początku planowania budowy, więc czas wykonania nie jest bardzo istotny,
uznaliśmy, że nasze rozwiązanie jest wystarczająco wydajne.
\subsection{Wyznaczanie dróg}
Wyznaczanie dróg z fabryki do miejsca budowy płotu zaimplementowaliśmy używając algorytmu Dijkstry.
\subsection{Zapisywanie melodii}
\subsubsection{Zamiana fragmentów}
aaaaaaaaaaaaaaaa
\subsubsection{Kompresja}
aaaaaaaaaaaaaaaa
\subsection{Patrole strażników}
aaaaaaaaaaaaaaaa
\printbibliography[heading=bibintoc]
\end{document}