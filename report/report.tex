\documentclass{article}

\usepackage[T1]{fontenc}
\usepackage[utf8]{inputenc}
\usepackage[polish]{babel}

\usepackage{csquotes}

\usepackage{hyperref}
\usepackage[capitalize,nameinlink]{cleveref}
\usepackage{url}

\usepackage[dvipsnames]{xcolor}
\hypersetup{
colorlinks=true,
linkcolor=BrickRed,
citecolor=Green,
urlcolor=blue,
frenchlinks=true,
pdftitle={Atak z trzeciego wymiaru},
pdfpagemode=FullScreen,
}

\usepackage[backend=biber, sorting=ynt]{biblatex}
\addbibresource{bibliography.bib}

\title{Atak z trzeciego wymiaru \\\large Sprawozdanie z projektu}
\author{Mikołaj Juda \and Mateusz Sobkowiak \and Paulina Grabowska}
\date{2024}

\begin{document}
\maketitle

\section{Architektura}
Problemy z wymagań projektu podzieliliśmy na podproblemy, w większości
rozwiązywalne istniejącymi algorytmami.
Rozwiązanie każdego podproblemu jest zaimplementowane jako biblioteka,
która jest wykorzystywana przez program zapewniający interfejs użytkownika
i minimalną logikę łączącą podproblemy. Pozwala to na proste testownaie
i umożliwa łatwe wprowadzenie wielu interfejsów, na przykład
tekstowy interfejs do użycia przez zewnętrzne programy i interfejs graficzny dla człowieka.

\section{Przepływ pracy}
Do organizacji pracy wykorzystywaliśmy narzędzia dostępne w serwisie Github.

\noindent Przydzielanie i śledzenie zadań realizowaliśmy na tablicy Kanban w GitHub Projects.

\noindent Kod rozwiązujący podproblemy był rozwijany na osobnych gałęziach,
i łączony z główną gałęzią używając pull requestów po recenzji przez wszystkich członków zespołu.
Używaliśmy też GitHub Actions do automatycznego uruchamiania lintera i testów.

\section{Rozwiązania problemów}

\subsection{Budowa płotu}

\subsubsection{Długość płotu}
Do wyznaczania otoczki wypukłaj wykorzystaliśmy algorytm Grahama.

\noindent Testy jednostkowe zawierają sprawdzenie poprawności wyniku dla przypadków krańcowych
i dla danych losowych w oparciu o sprawdzenie wypukłości i zawierania wszystkich punktów
używając pomocniczych funkcji.
\subsubsection{Parowanie tragarzy}
Parowanie tragarzy zrealizowaliśmy jako znajdowanie największego skojarzenia w grafie dwudzielnym.
Rozwiązanie zaimplementowaliśmy naiwnie używając algorytmu Edmondsa-Karpa ze względu na prostotę implementacji.
Przy złożoności \(\mathcal{O}(|V||E|)\)\cite{cs6820matchingnotes}
i fakcie, że algorytm uruchamiany jest tylko raz na początku planowania budowy,
uznaliśmy, że nasze rozwiązanie jest wystarczająco wydajne.

\noindent Testy jednostkowe sprawdzają czy wynik jest maksymalny,
gdyż nie ma prostej metody na zweryfikowanie czy jest największy
bez oblicznaia rozwiązania problemu.
\subsubsection{Wyznaczanie dróg}
Wyznaczanie najkrótszych dróg z fabryki do miejsca budowy płotu zaimplementowaliśmy używając algorytmu Dijkstry.

\noindent Testy jednostkowe sprawdzają poprawność wyniku
na podstawie porównania długości ścieżek z wcześniej wyznaczonymi poprawnymi wartościami
i przejściu wyznaczonymi ścieżkami sprawdzając prawidłowość.

\subsection{Zapisywanie melodii}
\subsubsection{Zamiana fragmentów}
Wyszukiwanie słów do zamiany w tekście zaimplementowaliśmy algorytmem Aho-Corasicka
ze względu na możliwość wyszukiwania wielu wzorców jednocześnie.

\noindent Testy jednostkowe sprawdzają poprawność wyniku
dla przypadków prostych, granicznych i losowych na podstawie porównania
z wynikiem algorytmu naiwnego.
\subsubsection{Kompresja}
Do kompresji wykorzystaliśmy kodowanie Huffmana.

\subsection{Patrole strażników}
Programowanie dynamiczne.

\printbibliography[heading=bibintoc]
\end{document}
